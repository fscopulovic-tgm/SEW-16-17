%!TEX root=../document.tex

\section{Aufgabe}
\subsection{Vorgangsweise}
Das, von dem Herr Professor Rafeiner-Magor, zur Verfügung gestellte \verb|pdf|-File\footnote{SEW\_4\_Events\_Bedingungsvariablen\_Queues\_Python.pdf}\space über \verb|Events|,\space\verb|Bedingungsvariablen| und \verb|Queues| war Anhaltspunkt für die Aufgabe. Durch dieses \verb|pdf|-File konnte ich mich sehr leicht an diese Aufgabe gewöhnen. Mit dem \verb|pdf|-File habe ich sofort begonnen meine Ideen zu Implementieren.
\subsection{Aufwand}
Der Aufwand war gering. Nachdem ich mich über \verb|Queues| informiert habe, war die Aufgabe leicht. Jedoch hatte ich Schwierigkeiten bei dem vollführen der erweiterten Aufgaben. Die Angabe hat mich verwirrt.
\subsection{Resultate}
Mein Resultat war ein \verb|txt|-File im Ordner \verb|files|. Dieses \verb|txt|-File ist gefühlt mit den ganzen Primzahlen, die gefunden wurden. Bei jedem neuen Aufruf wird das \verb|txt|-File überschrieben. Zudem wird in der Konsole außerdem die Primzahlen ausgegeben.
So stehen die Primzahlen im \verb|txt|-File.
\lstinputlisting[language=bash,caption=Primzahlen im txt-File]{code/primzahlen.txt}
So werden die Primzahlen in der Konsole ausgegeben.
\lstinputlisting[language=bash,caption=Ausgabe in der Konsole]{code/ausgabe.txt}
\subsection{Beobachtungen}
Ich fand es interessant wie es mit der \verb|Queue| funktionieren konnte und kein \verb|lock| oder Sonstiges benötitgt wurde.
\subsection{Schwierigkeiten}
Schwierigkeiten haben mir die Erweiterungen der Aufgabe bereitet. Hauptsächlich weil ich sie erst nach mehrfachen lesen verstanden habe. Sie waren recht unklar formuliert.
\subsection{Code}
\lstinputlisting[language=Python,caption=Verbraucher]{../consumer.py}\clearpage
\lstinputlisting[language=Python,caption=Erzeuger]{../producer.py}\clearpage
\lstinputlisting[language=Python,caption=Start]{../run_script.py}